% FORMAT AND PACKAGES - DOREMON DA O DAY
% {
\documentclass[a4paper]{article}
\usepackage{a4wide,amssymb,epsfig,latexsym,multicol,array,hhline,fancyhdr}
\usepackage{tcolorbox}
\usepackage{minted}
\usepackage{vntex}
\usepackage{amsmath}
\usepackage{lastpage}
\usepackage[lined,boxed,commentsnumbered]{algorithm2e}
\usepackage{enumerate}
\usepackage{xcolor}
\usepackage{graphicx}							% Standard graphics package
\usepackage{array}
\usepackage{tabularx, caption}
\usepackage{multirow}
\usepackage{multicol}
\usepackage{rotating}
\usepackage{graphics}
\usepackage{geometry}
\usepackage{setspace}
\usepackage{epsfig}
\usepackage{tikz}
\usepackage{xfrac}
\usepackage{bm}
\usepackage{biblatex}
\usepackage[colorlinks]{hyperref}
\newcommand{\cach}{\hspace*{1.5em}\ignorespaces}
% \usepackage[acronym,toc]{glossaries}
% \usepackage[symbols,nogroupskip,nonumberlist]{glossaries-extra}
\usepackage[
 sort=none,% no sorting or indexing required
 abbreviations,% create list of abbreviations
 symbols,% create list of symbols
 stylemods,style=list, % set the default glossary style
 nogroupskip, nonumberlist, nomain
]{glossaries-extra}


% FORMATTING
% {
\DeclareMathOperator{\arccot}{arccot}
\captionsetup[table]{name=Bảng}
\captionsetup[figure]{name=Hình}
\newenvironment{Description}{\list{}{%
    \let\makelabel\descriptionlabel    % this comes from the original description environment
    \setlength{\rightmargin}{\leftmargin}% this comes from the original quote environment
    \setlength{\labelwidth}{0pt}%          this is new
    }}{\endlist}

\addbibresource{citations.bib}
    
\hypersetup{urlcolor=blue,linkcolor=black,citecolor=black,colorlinks=true} 
\usetikzlibrary{arrows,snakes,backgrounds}
\definecolor{mathblue}{RGB}{0,114,188}
% \makeatletter  \def\m@th{\mathsurround\z@\color{mathblue}} \makeatother
% \everymath{\color{mathblue}}
% \setmathfont[Color=000000]{Arial}
%\usepackage{pstcol} 								% PSTricks with the standard color package
\newtheorem{theorem}{{\bf Theorem}}
\newtheorem{property}{{\bf Property}}
\newtheorem{proposition}{{\bf Proposition}}
\newtheorem{corollary}[proposition]{{\bf Corollary}}
\newtheorem{lemma}[proposition]{{\bf Lemma}}

\AtBeginDocument{\renewcommand{\listfigurename}{Danh sách hình ảnh}}
\AtBeginDocument{\renewcommand{\listtablename}{Danh sách bảng biểu}}
\AtBeginDocument{\renewcommand*\contentsname{Mục lục}}
\AtBeginDocument{\renewcommand*\refname{Tài liệu tham khảo}}
%\usepackage{fancyhdr}

\setlength{\headheight}{40pt}
\pagestyle{fancy}
\fancyhead{} % clear all header fields
\fancyhead[L]{
 \begin{tabular}{rl}
    \begin{picture}(25,15)(0,0)
    \put(0,-8){\includegraphics[width=8mm, height=8mm]{hcmut.png}}
    %\put(0,-8){\epsfig{width=10mm,figure=hcmut.eps}}
   \end{picture}&
	%\includegraphics[width=8mm, height=8mm]{hcmut.png} & %
	\begin{tabular}{l}
		\textbf{\bf \ttfamily Trường Đại học Bách Khoa TP.Hồ Chí Minh}\\
		\textbf{\bf \ttfamily Khoa Khoa học và Kỹ thuật máy tính}
	\end{tabular} 	
 \end{tabular}
}
\fancyhead[R]{
	\begin{tabular}{l}
		\tiny \bf \\
		\tiny \bf 
	\end{tabular}  }
\fancyfoot{} % clear all footer fields
\fancyfoot[L]{\scriptsize \ttfamily Báo cáo Bài tập lớn môn Mô hình hóa toán học - HK241 - Năm học 2024 - 2025}
\fancyfoot[R]{\scriptsize \ttfamily Trang {\thepage}/\pageref{LastPage}}
\renewcommand{\headrulewidth}{0.3pt}
\renewcommand{\footrulewidth}{0.3pt}

\setcounter{secnumdepth}{4}
\setcounter{tocdepth}{4}

\makeatletter
\newcounter {subsubsubsection}[subsubsection]
\renewcommand\thesubsubsubsection{\thesubsubsection .\@alph\c@subsubsubsection}
\newcommand\subsubsubsection{\@startsection{subsubsubsection}{4}{\z@}%
                                     {-3.25ex\@plus -1ex \@minus -.2ex}%
                                     {1.5ex \@plus .2ex}%
                                     {\normalfont\normalsize\bfseries}}
\newcommand*\l@subsubsubsection{\@dottedtocline{3}{10.0em}{4.1em}}
\newcommand*{\subsubsubsectionmark}[1]{}
% \def\m@th{\mathsurround\z@\color{mathblue}}
\makeatother
% }
% }

% ACRONYMS & SYMBOLS
% {
% \makeglossaries
\setabbreviationstyle{long-short}
\newabbreviation{csp}{CSP}{Cutting Stock Problem}
\newabbreviation{ffd}{FFD}{First Fit Decreasing}
\newabbreviation{ga}{GA}{Genetic Algorithm}
\newabbreviation{lp}{LP}{Linear Programming}
% \glsnoexpandfields
\glsxtrnewsymbol[description = {Tập hợp số tự nhiên}]{natural}{\ensuremath{\mathbb{N}}}

% }
%
% DOCUMENT
\begin{document}

% TITLE PAGE
\begin{titlepage}
\begin{center}
ĐẠI HỌC QUỐC GIA THÀNH PHỐ HỒ CHÍ MINH\\
TRƯỜNG ĐẠI HỌC BÁCH KHOA\\
KHOA KHOA HỌC VÀ KỸ THUẬT MÁY TÍNH\\
\end{center}

\vspace{1cm}

\begin{figure}[h!]
\begin{center}
\includegraphics[width=3cm]{hcmut.png}
\end{center}
\end{figure}

\vspace{1cm}


\begin{center}
\begin{tabular}{c}
\multicolumn{1}{c}{\textbf{{\Large MÔ HÌNH HÓA TOÁN HỌC (CO2011)}}}\\
~~\\
\hline
\\
\multicolumn{1}{l}{\textbf{{\Large Bài tập lớn}}}\\
\\
\textbf{\textit{{\Huge abc}}}\\
\\
\hline
\end{tabular}
\end{center}

\vspace{2cm}

\begin{table}[h]
\centering
    \begin{tabular}{rl}
    \hspace{3 cm}\textbf{GVHD}:
    & tao\\

    & \\[10pt]
\textbf{Sinh viên}: & Lư Chấn Vũ - 2313955 \emph{(Lớp L04 - Nhóm 122, \textbf{Leader})} \\
& Vũ Minh Sang - 2312944 \emph{(Lớp L04 - Nhóm 122)} \\
& Nguyễn Quang Huy - 2311202 \emph{(Lớp L01 - Nhóm 122)} \\
& Lê Minh Khoa - 2311593 \emph{(Lớp L04 - Nhóm 122)} \\
& Lê Minh Trí - 2313593 \emph{(Lớp L04 - Nhóm 122)} \\
    \end{tabular}
\end{table}

\begin{center}
{\footnotesize TP. HỒ CHÍ MINH, 12/2024}
\end{center}
\end{titlepage}

\pagebreak
\tableofcontents

\pagebreak

% Glossaries
% {}
\printunsrtglossary[type={symbols}, title={Danh sách kí hiệu}]
\printunsrtglossary[type={abbreviations}, title={Danh sách từ viết tắt}]
\pagebreak
\listoffigures
\listoftables
\pagebreak
\addcontentsline{toc}{section}{\listfigurename}
\addcontentsline{toc}{section}{\listtablename}

% }

% Member list
\section*{Danh sách thành viên và nhiệm vụ}
\addcontentsline{toc}{section}{Danh sách thành viên và nhiệm vụ}
\begin{center}
\begin{table}[H]
\centering
\begin{tabular}{|c|c|c|l|c|}
\hline
\textbf{STT} & \textbf{Họ và tên} & \textbf{MSSV} & \textbf{Nhiệm vụ} & \textbf{\% hoàn thành}\\
\hline 
%%%%%Student 1%%%%%%%%%%
\multirow{3}{*}{1} & \multirow{3}{*}{Lư Chấn Vũ} & \multirow{3}{*}{2313955} & 
- Code: GA. & \multirow{3}{*}{100\%}\\
 & &  & - Báo cáo: Mục 5.3. & \\
\hline
%%%%%Student 2%%%%%%%%%%
\multirow{3}{*}{2} & \multirow{3}{*}{Vũ Minh Sang} & \multirow{3}{*}{2312944} & 
- Báo cáo: Mục 3, 5.1. & \multirow{3}{*}{100\%}\\
 & &  & - Tổng hợp và chỉnh sửa báo cáo.  & \\
\hline
%%%%%Student 3%%%%%%%%%%
\multirow{3}{*}{3} & \multirow{3}{*}{Nguyễn Quang Huy} & \multirow{3}{*}{2311202} & 
- Code: FFD. & \multirow{3}{*}{100\%}\\
 & &  & - Báo cáo: Mục 4.1, 5.2. & \\
\hline
%%%%%Student 4%%%%%%%%%%
\multirow{3}{*}{4} & \multirow{3}{*}{Lê Minh Khoa} & \multirow{3}{*}{2311593} & 
- Code: GA. & \multirow{3}{*}{100\%}\\
 & &  & - Báo cáo: Mục 2, 4.2. & \\
\hline
%%%%%Student 5%%%%%%%%%%
\multirow{3}{*}{5} & \multirow{3}{*}{Lê Minh Trí} & \multirow{3}{*}{2313593} & 
- Code: FFD. & \multirow{3}{*}{100\%}\\
 & &  & - Báo cáo: Mục 1, 6, 7.& \\
\hline
\end{tabular}
\caption{\label{table1}Danh sách thành viên và nhiệm vụ}
\end{table}
\end{center}

\pagebreak

% for testing
% REFERENCES
\pagebreak
\nocite{*}
\printbibliography[
heading=bibintoc,
title={References}
]
\begin{thebibliography}{9}

\bibitem{gilmore1961}
Gilmore, P.C. and Gomory, R.E., 1961. A linear programming approach to the cutting stock problem. \textit{Operations Research}, 9(6), pp.849-859.

\bibitem{dyckhoff1991}
Dyckhoff, H., 1991. Cutting stock problems and solution procedures. \textit{European Journal of Operational Research}, 44(2), pp.145-159.

\bibitem{cui2000}
Cui, Z. and Zhang, L., 2000. A near-optimal solution to a two-dimensional cutting stock problem. \textit{Journal of Optimization Theory and Applications}, 107(2), pp.393-408.

\bibitem{bennell2021}
Bennell, J.A., Oliveira, J.F. and Hitchen, M., 2021. Exact solution techniques for two-dimensional cutting and packing. \textit{European Journal of Operational Research}, 293(3), pp.949-963.

\bibitem{lodi2021}
Lodi, A., Martello, S. and Vigo, D., 2021. A heuristic approach for two-dimensional rectangular cutting. \textit{Computers \& Operations Research}, 127, p.105121.

\bibitem{silva2023}
Silva, R.A., Pinto, T.L. and Gomes, C.J., 2023. Approximation method for solving two-dimensional cutting stock problems. \textit{Mathematical Programming}, 150(1), pp.195-220.

\bibitem{vanderbeck1996}
Vanderbeck, F. and Wolsey, L.A., 1996. An exact algorithm for the two-dimensional cutting stock problem. \textit{Computational Optimization and Applications}, 3(1), pp.123-143.

\bibitem{alvarez2005}
Alvarez-Valdes, R., Parajon, A. and Tamarit, J.M., 2005. A tabu search algorithm for large-scale two-dimensional cutting stock problems. \textit{Computers \& Operations Research}, 32(5), pp.985-1007.

\bibitem{beasley1985}
Beasley, J.E., 1985. An exact two-dimensional non-guillotine cutting tree search procedure. \textit{Operations Research}, 33(1), pp.49-64.

\bibitem{martello1990}
Martello, S. and Toth, P., 1990. Knapsack Problems: Algorithms and Computer Implementations. Wiley-Interscience.

\bibitem{gfgGreedyAlgorithm}
GeeksforGeeks, 2024. Introduction to Greedy Algorithm - Data Structures and Algorithm Tutorials. Available at: \url{https://www.geeksforgeeks.org/introduction-to-greedy-algorithm-data-structures-and-algorithm-tutorials/} [Accessed 25 Nov. 2024].

\bibitem{orlibrary}
OR-Library, Cutting Stock Problem Data. Available at: \url{http://people.brunel.ac.uk/~mastjjb/jeb/orlib/cutinfo.html} [Accessed 25 Nov. 2024].

\bibitem{dyckhoff1990}
Dyckhoff, H., 1990. A typology of cutting and packing problems. \textit{European Journal of Operational Research}, 44(2), pp.145-159.

\end{thebibliography}
\end{document}