% FORMAT AND PACKAGES - DOREMON DA O DAY
% {
\documentclass[a4paper]{article}
\usepackage{a4wide,amssymb,epsfig,latexsym,multicol,array,hhline,fancyhdr}
\usepackage{tcolorbox}
\usepackage{minted}
\usepackage{vntex}
\usepackage{amsmath}
\usepackage{lastpage}
\usepackage[lined,boxed,commentsnumbered]{algorithm2e}
\usepackage{enumerate}
\usepackage{xcolor}
\usepackage{graphicx}							% Standard graphics package
\usepackage{array}
\usepackage{tabularx, caption}
\usepackage{multirow}
\usepackage{multicol}
\usepackage{rotating}
\usepackage{graphics}
\usepackage{geometry}
\usepackage{setspace}
\usepackage{epsfig}
\usepackage{tikz}
\usepackage{xfrac}
\usepackage{bm}
\usepackage{biblatex}
\usepackage[colorlinks]{hyperref}
\newcommand{\cach}{\hspace*{1.5em}\ignorespaces}
% \usepackage[acronym,toc]{glossaries}
% \usepackage[symbols,nogroupskip,nonumberlist]{glossaries-extra}
\usepackage[
 sort=none,% no sorting or indexing required
 abbreviations,% create list of abbreviations
 symbols,% create list of symbols
 stylemods,style=list, % set the default glossary style
 nogroupskip, nonumberlist, nomain
]{glossaries-extra}


% FORMATTING
% {
\DeclareMathOperator{\arccot}{arccot}
\captionsetup[table]{name=Bảng}
\captionsetup[figure]{name=Hình}
\newenvironment{Description}{\list{}{%
    \let\makelabel\descriptionlabel    % this comes from the original description environment
    \setlength{\rightmargin}{\leftmargin}% this comes from the original quote environment
    \setlength{\labelwidth}{0pt}%          this is new
    }}{\endlist}

\addbibresource{citations.bib}
    
\hypersetup{urlcolor=blue,linkcolor=black,citecolor=black,colorlinks=true} 
\usetikzlibrary{arrows,snakes,backgrounds}
\definecolor{mathblue}{RGB}{0,114,188}
% \makeatletter  \def\m@th{\mathsurround\z@\color{mathblue}} \makeatother
% \everymath{\color{mathblue}}
% \setmathfont[Color=000000]{Arial}
%\usepackage{pstcol} 								% PSTricks with the standard color package
\newtheorem{theorem}{{\bf Theorem}}
\newtheorem{property}{{\bf Property}}
\newtheorem{proposition}{{\bf Proposition}}
\newtheorem{corollary}[proposition]{{\bf Corollary}}
\newtheorem{lemma}[proposition]{{\bf Lemma}}

\AtBeginDocument{\renewcommand{\listfigurename}{Danh sách hình ảnh}}
\AtBeginDocument{\renewcommand{\listtablename}{Danh sách bảng biểu}}
\AtBeginDocument{\renewcommand*\contentsname{Mục lục}}
\AtBeginDocument{\renewcommand*\refname{Tài liệu tham khảo}}
%\usepackage{fancyhdr}

\setlength{\headheight}{40pt}
\pagestyle{fancy}
\fancyhead{} % clear all header fields
\fancyhead[L]{
 \begin{tabular}{rl}
    \begin{picture}(25,15)(0,0)
    \put(0,-8){\includegraphics[width=8mm, height=8mm]{hcmut.png}}
    %\put(0,-8){\epsfig{width=10mm,figure=hcmut.eps}}
   \end{picture}&
	%\includegraphics[width=8mm, height=8mm]{hcmut.png} & %
	\begin{tabular}{l}
		\textbf{\bf \ttfamily Trường Đại học Bách Khoa TP.Hồ Chí Minh}\\
		\textbf{\bf \ttfamily Khoa Khoa học và Kỹ thuật máy tính}
	\end{tabular} 	
 \end{tabular}
}
\fancyhead[R]{
	\begin{tabular}{l}
		\tiny \bf \\
		\tiny \bf 
	\end{tabular}  }
\fancyfoot{} % clear all footer fields
\fancyfoot[L]{\scriptsize \ttfamily Báo cáo Bài tập lớn môn Mô hình hóa toán học - HK241 - Năm học 2024 - 2025}
\fancyfoot[R]{\scriptsize \ttfamily Trang {\thepage}/\pageref{LastPage}}
\renewcommand{\headrulewidth}{0.3pt}
\renewcommand{\footrulewidth}{0.3pt}

\setcounter{secnumdepth}{4}
\setcounter{tocdepth}{4}

\makeatletter
\newcounter {subsubsubsection}[subsubsection]
\renewcommand\thesubsubsubsection{\thesubsubsection .\@alph\c@subsubsubsection}
\newcommand\subsubsubsection{\@startsection{subsubsubsection}{4}{\z@}%
                                     {-3.25ex\@plus -1ex \@minus -.2ex}%
                                     {1.5ex \@plus .2ex}%
                                     {\normalfont\normalsize\bfseries}}
\newcommand*\l@subsubsubsection{\@dottedtocline{3}{10.0em}{4.1em}}
\newcommand*{\subsubsubsectionmark}[1]{}
% \def\m@th{\mathsurround\z@\color{mathblue}}
\makeatother
% }
% }

% ACRONYMS & SYMBOLS
% {
% \makeglossaries
\setabbreviationstyle{long-short}
\newabbreviation{csp}{CSP}{Cutting Stock Problem}
\newabbreviation{ffd}{FFD}{First Fit Decreasing}
\newabbreviation{ga}{GA}{Genetic Algorithm}
\newabbreviation{lp}{LP}{Linear Programming}
% \glsnoexpandfields
\glsxtrnewsymbol[description = {Tập hợp số tự nhiên}]{natural}{\ensuremath{\mathbb{N}}}

% }
%
% DOCUMENT
\begin{document}

% TITLE PAGE
\begin{titlepage}
\begin{center}
ĐẠI HỌC QUỐC GIA THÀNH PHỐ HỒ CHÍ MINH\\
TRƯỜNG ĐẠI HỌC BÁCH KHOA\\
KHOA KHOA HỌC VÀ KỸ THUẬT MÁY TÍNH\\
\end{center}

\vspace{1cm}

\begin{figure}[h!]
\begin{center}
\includegraphics[width=3cm]{hcmut.png}
\end{center}
\end{figure}

\vspace{1cm}


\begin{center}
\begin{tabular}{c}
\multicolumn{1}{c}{\textbf{{\Large Đồ án tổng hợp - CNPM (CO3103) }}}\\
~~\\
\hline
\\
\multicolumn{1}{l}{\textbf{{\Large Báo cáo }}}\\
\\
\textbf{\textit{{\Huge Ứng dụng nghe nhạc BKSound}}}\\
\\
\hline
\end{tabular}
\end{center}

\vspace{2cm}

\begin{table}[h]
\centering
    \begin{tabular}{rl}
    \hspace{3 cm}\textbf{GVHD}:
    & Trần Trương Tuấn Phát\\

    & \\[10pt]
\textbf{Sinh viên}: & Lư Chấn Vũ - 2313955 \emph{(Lớp L04 - Nhóm 122, \textbf{Leader})} \\
& Vũ Minh Sang - 2312944 \emph{(Lớp L04 - Nhóm 122)} \\
& Nguyễn Quang Huy - 2311202 \emph{(Lớp L01 - Nhóm 122)} \\
& Lê Minh Khoa - 2311593 \emph{(Lớp L04 - Nhóm 122)} \\
& Lê Minh Trí - 2313593 \emph{(Lớp L04 - Nhóm 122)} \\
    \end{tabular}
\end{table}

\begin{center}
{\footnotesize TP. HỒ CHÍ MINH, 12/2024}
\end{center}
\end{titlepage}

\pagebreak
\tableofcontents

\pagebreak

% Glossaries
% {}
\printunsrtglossary[type={symbols}, title={Danh sách kí hiệu}]
\printunsrtglossary[type={abbreviations}, title={Danh sách từ viết tắt}]
\pagebreak
\listoffigures
\listoftables
\pagebreak
\addcontentsline{toc}{section}{\listfigurename}
\addcontentsline{toc}{section}{\listtablename}

% }

% Member list
\section*{Danh sách thành viên và nhiệm vụ}
\addcontentsline{toc}{section}{Danh sách thành viên và nhiệm vụ}
\begin{center}
\begin{table}[H]
\centering
\begin{tabular}{|c|c|c|l|c|}
\hline
\textbf{STT} & \textbf{Họ và tên} & \textbf{MSSV} & \textbf{Nhiệm vụ} & \textbf{\% hoàn thành}\\
\hline 
%%%%%Student 1%%%%%%%%%%
\multirow{3}{*}{1} & \multirow{3}{*}{Lư Chấn Vũ} & \multirow{3}{*}{2313955} & 
- Code: GA. & \multirow{3}{*}{100\%}\\
 & &  & - Báo cáo: Mục 5.3. & \\
\hline
%%%%%Student 2%%%%%%%%%%
\multirow{3}{*}{2} & \multirow{3}{*}{Vũ Minh Sang} & \multirow{3}{*}{2312944} & 
- Báo cáo: Mục 3, 5.1. & \multirow{3}{*}{100\%}\\
 & &  & - Tổng hợp và chỉnh sửa báo cáo.  & \\
\hline
%%%%%Student 3%%%%%%%%%%
\multirow{3}{*}{3} & \multirow{3}{*}{Nguyễn Quang Huy} & \multirow{3}{*}{2311202} & 
- Code: FFD. & \multirow{3}{*}{100\%}\\
 & &  & - Báo cáo: Mục 4.1, 5.2. & \\
\hline
%%%%%Student 4%%%%%%%%%%
\multirow{3}{*}{4} & \multirow{3}{*}{Lê Minh Khoa} & \multirow{3}{*}{2311593} & 
- Code: GA. & \multirow{3}{*}{100\%}\\
 & &  & - Báo cáo: Mục 2, 4.2. & \\
\hline
%%%%%Student 5%%%%%%%%%%
\multirow{3}{*}{5} & \multirow{3}{*}{Lê Minh Trí} & \multirow{3}{*}{2313593} & 
- Code: FFD. & \multirow{3}{*}{100\%}\\
 & &  & - Báo cáo: Mục 1, 6, 7.& \\
\hline
\end{tabular}
\caption{\label{table1}Danh sách thành viên và nhiệm vụ}
\end{table}
\end{center}

\pagebreak
\section{Tổng quan về đề tài}
\subsection{Mô tả đề tài}
\subsection{Mục tiêu đề tài}
\subsection{Phạm vi đề tài}
\section{Functional Requirements - Non-functional Requirements}
\subsection{Functional Requirements}
\subsubsection{Non-interactive Requirements}
\subsubsubsection{Tính năng Streaming ở nhiều mức bitrate}

\cach Chức năng này được thiết kế nhằm tối ưu trải nghiệm nghe nhạc của Người dùng trên nhiều loại thiết bị và trong các điều kiện mạng khác nhau. 
Sau khi một bài hát được upload thành công lên hệ thống, máy chủ sẽ tự động tiến hành xử lý và chuyển đổi file gốc sang nhiều phiên bản với các mức bitrate khác nhau (ví dụ: 64kbps, 128kbps, 320kbps). 

\begin{itemize}
   \item \textbf{Cách hoạt động:}
\begin{itemize}
    \item Khi Người dùng phát một bài hát, hệ thống sẽ cung cấp nhiều lựa chọn bitrate khác nhau (64kbps, 128kbps, 320kbps).
    \begin{itemize}
        \item 64kbps: Phù hợp cho kết nối mạng yếu, tiết kiệm dữ liệu.
        \item 128kbps: Chất lượng tiêu chuẩn, cân bằng giữa tốc độ tải và chất lượng âm thanh.
        \item 320kbps: Chất lượng cao, dành cho Người dùng muốn trải nghiệm âm nhạc tốt nhất.
    \end{itemize}
    \item Người dùng có thể chọn thủ công mức bitrate mong muốn, phù hợp với chất lượng mạng và nhu cầu nghe nhạc.
    \item Hệ thống hỗ trợ \textbf{Adaptive Streaming}: trong quá trình nghe, nếu mạng yếu hoặc không ổn định, bitrate sẽ tự động hạ xuống để tránh gián đoạn; khi mạng mạnh trở lại, bitrate được nâng lên để đảm bảo chất lượng âm thanh tốt nhất.
    \item Các phiên bản nhạc ở nhiều bitrate được tạo sẵn trong quá trình xử lý upload, do đó việc chuyển đổi diễn ra nhanh chóng và mượt mà.
\end{itemize}
\item \textbf{Input:}
\begin{itemize}
    \item File nhạc gốc (định dạng hợp lệ: MP3, WAV, FLAC,...).
    \item Metadata bài hát (tên, nghệ sĩ, thể loại, ảnh bìa).
    \item Thông tin cấu hình hệ thống (các mức bitrate cần tạo).
\end{itemize}

\item \textbf{Output:}
\begin{itemize}
    \item Các phiên bản bài hát ở nhiều mức bitrate (64kbps, 128kbps, 320kbps).
    \item Đường dẫn hoặc ID truy cập các file đã xử lý để phát trực tuyến.
\end{itemize}


\item \textbf{Lợi ích:}
\begin{itemize}
    \item Người dùng có trải nghiệm nghe nhạc mượt mà, ngay cả khi mạng yếu.
    \item Tối ưu dung lượng lưu trữ và băng thông cho hệ thống.
    \item Đáp ứng nhu cầu đa dạng: nghe nhạc tiết kiệm dữ liệu hoặc chất lượng cao.
\end{itemize}
\end{itemize}
\subsubsubsection{Tính năng gợi ý bài hát dựa vào lượt nghe gần đây}
\cach Chức năng này giúp cá nhân hoá trải nghiệm nghe nhạc của Người dùng. 
Hệ thống sẽ phân tích lịch sử nghe nhạc gần đây (bài hát, nghệ sĩ, thể loại) để tự động đưa ra danh sách gợi ý phù hợp với sở thích hiện tại của Người dùng. 
Danh sách gợi ý có thể hiển thị dưới dạng playlist hoặc phần “Đề xuất cho bạn” trên giao diện chính.
\begin{itemize}
    \item \textbf{Cách hoạt động:}
    \begin{itemize}
        \item Hệ thống lưu trữ và theo dõi lịch sử nghe nhạc của Người dùng.
        \item Dựa trên dữ liệu lượt nghe gần đây, hệ thống áp dụng thuật toán gợi ý (lọc cộng tác, phân tích nội dung hoặc kết hợp).
        \item Trả về danh sách bài hát, album hoặc Nghệ sĩ có mức độ tương đồng cao.
    \end{itemize}

    \item \textbf{Input:}
    \begin{itemize}
        \item Lịch sử nghe nhạc gần đây (danh sách bài hát đã phát).
        \item Metadata của bài hát (thể loại, nghệ sĩ, album, nhãn).
        \item Dữ liệu hành vi Người dùng khác (để tăng độ chính xác).
    \end{itemize}

    \item \textbf{Output:}
    \begin{itemize}
        \item Danh sách gợi ý gồm các bài hát, playlist hoặc Nghệ sĩ liên quan.
        \item Giao diện hiển thị mục “Gợi ý cho bạn” được cập nhật động.
    \end{itemize}

    \item \textbf{Lợi ích:}
    \begin{itemize}
        \item Giúp Người dùng khám phá thêm nhạc mới phù hợp với sở thích.
        \item Tăng mức độ gắn bó và thời gian sử dụng ứng dụng.
        \item Nâng cao trải nghiệm nhờ cá nhân hoá thông minh.
    \end{itemize}
\end{itemize}

\subsubsubsection{Tính năng tự động tạo ảnh cover/thumbnail nếu Người dùng không upload}
\cach Khi Người dùng upload bài hát nhưng không cung cấp ảnh bìa (cover) hoặc thumbnail, hệ thống sẽ tự động sinh ra hình ảnh thay thế. 
Hình ảnh được tạo có thể dựa trên thông tin metadata của bài hát (tên, nghệ sĩ, thể loại) hoặc sử dụng mẫu có sẵn. 
Mục tiêu là đảm bảo tất cả các bài hát trong hệ thống đều có ảnh hiển thị nhất quán và trực quan.

\begin{itemize}
    \item \textbf{Cách hoạt động:}
    \begin{itemize}
        \item Hệ thống kiểm tra file ảnh cover kèm theo khi upload bài hát.
        \item Nếu không có ảnh, hệ thống kích hoạt module sinh ảnh tự động.
        \item Ảnh được tạo bằng cách:
        \begin{itemize}
            \item Sinh ngẫu nhiên từ template mặc định theo thể loại nhạc.
            \item Kết hợp text (tên bài hát, nghệ sĩ) với nền gradient hoặc ảnh mẫu.
        \end{itemize}
        \item Ảnh được gán cho bài hát và hiển thị trong thư viện, playlist, cũng như trình phát nhạc.
    \end{itemize}

    \item \textbf{Input:}
    \begin{itemize}
        \item File nhạc gốc upload lên hệ thống.
        \item Metadata bài hát (tên, nghệ sĩ, thể loại).
        \item Bộ template mặc định của hệ thống.
    \end{itemize}

    \item \textbf{Output:}
    \begin{itemize}
        \item Ảnh cover/thumbnail tự động sinh ra cho bài hát.
        \item Đường dẫn hoặc ID của ảnh lưu trữ trên server.
    \end{itemize}


    \item \textbf{Lợi ích:}
    \begin{itemize}
        \item Đảm bảo giao diện ứng dụng đồng bộ, không có bài hát bị thiếu ảnh hiển thị.
        \item Giúp Người dùng tiết kiệm thời gian chuẩn bị file ảnh trước khi upload.
        \item Nâng cao trải nghiệm nghe nhạc với hình ảnh trực quan và thẩm mỹ.
    \end{itemize}
\end{itemize}
\subsubsubsection{Tính năng tự động phát bài hát tiếp theo trong playlist/queue}  
\cach Khi một bài hát trong playlist hoặc queue kết thúc, hệ thống sẽ tự động phát bài hát kế tiếp mà không cần thao tác thủ công từ Người dùng.  
Điều này giúp trải nghiệm nghe nhạc liền mạch và thuận tiện, đặc biệt khi Người dùng đang nghe nhạc trong lúc làm việc, tập luyện hoặc thư giãn.  

\begin{itemize}
    \item \textbf{Cách hoạt động:}
    \begin{itemize}
        \item Khi bài hát hiện tại kết thúc, hệ thống kiểm tra danh sách playlist/queue đang hoạt động.
        \item Nếu còn bài hát trong danh sách, hệ thống sẽ tự động phát bài kế tiếp theo thứ tự.
        \item Nếu đến cuối danh sách: Tùy chế độ, có thể dừng phát nhạc, lặp lại playlist, hoặc bật chế độ phát ngẫu nhiên.
        \item Quá trình chuyển bài diễn ra mượt mà, không tạo khoảng trống âm thanh lớn.
    \end{itemize}

    \item \textbf{Input:}
    \begin{itemize}
        \item Danh sách playlist hoặc queue do Người dùng tạo hoặc hệ thống gợi ý.
        \item Thiết lập chế độ phát nhạc (bình thường, lặp lại, ngẫu nhiên).
    \end{itemize}

    \item \textbf{Output:}
    \begin{itemize}
        \item Bài hát kế tiếp được phát tự động ngay sau khi bài hiện tại kết thúc.
        \item Trạng thái phát nhạc được cập nhật trong trình phát và UI của ứng dụng.
    \end{itemize}


    \item \textbf{Lợi ích:}
    \begin{itemize}
        \item Mang lại trải nghiệm nghe nhạc liền mạch, không bị gián đoạn.
        \item Người dùng không cần thao tác thủ công, thuận tiện trong nhiều ngữ cảnh.
        \item Hỗ trợ các chế độ phát linh hoạt, phù hợp với nhu cầu nghe nhạc đa dạng.
    \end{itemize}
\end{itemize}
\subsubsubsection{Tính năng thông báo khi Nghệ sĩ được follow phát hành bài hát/album mới}  
\cach Khi Người dùng follow một Nghệ sĩ, hệ thống sẽ theo dõi các hoạt động phát hành của Nghệ sĩ đó.  
Khi có bài hát hoặc album mới được phát hành, hệ thống sẽ gửi thông báo đến Người dùng để họ có thể thưởng thức ngay lập tức.  
Điều này giúp tăng mức độ gắn kết giữa Người dùng và Nghệ sĩ, đồng thời nâng cao trải nghiệm khám phá nhạc mới.  

\begin{itemize}
    \item \textbf{Cách hoạt động:}
    \begin{itemize}
        \item Hệ thống lưu danh sách Nghệ sĩ mà Người dùng đã follow.
        \item Khi Nghệ sĩ phát hành bài hát/album mới, hệ thống kiểm tra và xác định những Người dùng đã follow.
        \item Gửi thông báo qua ứng dụng (push notification).
        \item Thông báo chứa thông tin cơ bản như tên bài hát/album, ngày phát hành, và liên kết để nghe trực tiếp.
    \end{itemize}

    \item \textbf{Input:}
    \begin{itemize}
        \item Danh sách Nghệ sĩ được Người dùng follow.
        \item Dữ liệu phát hành bài hát/album mới của Nghệ sĩ.
    \end{itemize}

    \item \textbf{Output:}
    \begin{itemize}
        \item Thông báo gửi đến Người dùng khi có bài hát/album mới.
        \item Liên kết dẫn trực tiếp đến nội dung nhạc trong ứng dụng.
    \end{itemize}

    \item \textbf{Lợi ích:}
    \begin{itemize}
        \item Người dùng luôn cập nhật kịp thời nhạc mới từ Nghệ sĩ yêu thích.
        \item Tăng mức độ tương tác và gắn bó giữa Người dùng và nền tảng.
        \item Giúp Nghệ sĩ tiếp cận nhanh chóng đến fan hâm mộ của mình.
    \end{itemize}
\end{itemize}
\subsubsubsection{Tính năng thống kê thời gian nghe và lượt nghe}  
\cach Hệ thống cung cấp cho Người dùng thống kê chi tiết về hoạt động nghe nhạc của họ, bao gồm tổng thời gian nghe, số lượt nghe theo ngày/tuần/tháng, các bài hát được nghe nhiều nhất, và Nghệ sĩ được yêu thích nhất.  
Mục tiêu là giúp Người dùng theo dõi thói quen nghe nhạc, đồng thời tăng mức độ gắn bó với ứng dụng thông qua các báo cáo cá nhân hóa.  

\begin{itemize}
    \item \textbf{Cách hoạt động:}
    \begin{itemize}
        \item Mỗi lần Người dùng phát một bài hát, hệ thống ghi nhận thời lượng nghe và số lượt nghe.
        \item Dữ liệu được lưu trữ và cập nhật liên tục trong cơ sở dữ liệu.
        \item Người dùng có thể xem thống kê dưới dạng biểu đồ, danh sách hoặc báo cáo theo mốc thời gian (ngày/tuần/tháng/năm).
        \item Hệ thống có thể tạo báo cáo đặc biệt (ví dụ: tổng kết cuối năm) để tăng trải nghiệm Người dùng.
    \end{itemize}

    \item \textbf{Input:}
    \begin{itemize}
        \item Hoạt động nghe nhạc của Người dùng (thời gian phát, bài hát, Nghệ sĩ, playlist).
        \item Thông tin Người dùng để liên kết dữ liệu thống kê.
    \end{itemize}

    \item \textbf{Output:}
    \begin{itemize}
        \item Báo cáo thống kê: tổng thời gian nghe, số lượt nghe, top bài hát/Nghệ sĩ/album.
        \item Biểu đồ hoặc bảng dữ liệu trực quan hiển thị trong ứng dụng.
    \end{itemize}


    \item \textbf{Lợi ích:}
    \begin{itemize}
        \item Người dùng có thể theo dõi thói quen nghe nhạc của bản thân.
        \item Tăng sự gắn bó với ứng dụng thông qua báo cáo cá nhân hóa.
        \item Tạo cơ sở dữ liệu cho hệ thống gợi ý nhạc chính xác hơn.
        \item Có thể sử dụng để tổ chức sự kiện đặc biệt (ví dụ: “Top bài hát năm của bạn”).
    \end{itemize}
\end{itemize}
\subsubsubsection{Tính năng tạo và đồng bộ lời bài hát (Lyric Sync)}  
\cach Hệ thống cho phép Người dùng trải nghiệm nghe nhạc với phần hiển thị lời bài hát được đồng bộ theo thời gian phát nhạc (karaoke-style).  
Khi upload bài hát, Nghệ sĩ hoặc quản trị viên có thể thêm file lyric kèm theo, hoặc hệ thống hỗ trợ nhập thủ công và đồng bộ từng câu hát với thời gian.  
Mục tiêu là mang đến trải nghiệm nghe nhạc sinh động, giúp Người dùng dễ dàng theo dõi và hát theo bài hát.  

\begin{itemize}
    \item \textbf{Cách hoạt động:}
    \begin{itemize}
        \item Khi upload, Người dùng hoặc Nghệ sĩ cung cấp file lời bài hát (.lrc hoặc định dạng hỗ trợ).
        \item Nếu không có file sẵn, hệ thống cho phép nhập lời và sử dụng công cụ đồng bộ thời gian (timestamp editor).
        \item Khi phát nhạc, trình phát hiển thị lời bài hát, cuộn và highlight theo đúng nhịp bài hát.
    \end{itemize}

    \item \textbf{Input:}
    \begin{itemize}
        \item File nhạc upload lên hệ thống.
        \item File lyric (.lrc) hoặc text lyric do Người dùng/Nghệ sĩ nhập vào.
        \item Timestamp đồng bộ (có thể nhập thủ công hoặc tự động gợi ý).
    \end{itemize}

    \item \textbf{Output:}
    \begin{itemize}
        \item Lời bài hát hiển thị trực quan và đồng bộ theo thời gian phát nhạc.
        \item Highlight từng dòng lyric theo nhạc.
    \end{itemize}

    \item \textbf{Lợi ích:}
    \begin{itemize}
        \item Nâng cao trải nghiệm nghe nhạc, đặc biệt với Người dùng muốn hát theo.
        \item Tăng tính chuyên nghiệp, tạo cảm giác gần gũi như ứng dụng karaoke.
    \end{itemize}
\end{itemize}

\subsection{Non-functional Requirements}
\section{Use-case Diagram và Use-case Scenario}
\begin{table}[h!]
\centering
\renewcommand{\arraystretch}{1.3} % tăng khoảng cách dòng trong bảng
\begin{tabularx}{\textwidth}{|l|X|}
\hline
\textbf{Use case name} & Upload bài hát/album \\ \hline
\textbf{Actors}        & Nghệ sĩ \\ \hline
\textbf{Description}   & Chức năng cho phép Người dùng tải lên các bài hát của mình lên hệ thống để lưu trữ, quản lý và chia sẻ với cộng đồng nghe nhạc. \\ \hline
\textbf{Trigger}       & Người dùng chọn chức năng “Upload nhạc” trên giao diện hệ thống sau khi đăng nhập thành công. \\ \hline
\textbf{Pre-Condition(s)} 
& 1. Người dùng đã đăng nhập vào hệ thống. \newline
  2. Thiết bị của Người dùng được kết nối internet. \newline
  3. File nhạc đáp ứng đúng định dạng và dung lượng cho phép (ví dụ: MP3, WAV, dung lượng tối đa 20MB). \\ \hline
\textbf{Post-Condition(s)} 
& Bài hát được lưu trữ trong hệ thống, hiển thị trong thư viện cá nhân và có thể phát trực tuyến cho Người dùng khác (nếu được chia sẻ công khai). \\ \hline
\textbf{Normal Flow}   
& 1. Người dùng chọn vào phần “Upload nhạc”. \newline
  2. Hệ thống hiển thị form tải nhạc (chọn file nhạc, nhập tiêu đề, Nghệ sĩ, thể loại, ảnh bìa,...). \newline
  3. Người dùng nhập thông tin và chọn file nhạc từ thiết bị. \newline
  4. Người dùng nhấn nút “Upload”. \newline
  5. Hệ thống tiến hành tải file nhạc lên server. \newline
  6. Sau khi upload thành công, hệ thống thông báo và hiển thị bài hát trong thư viện cá nhân. \\ \hline
\textbf{Exception Flow} 
& 3a. Nếu file nhạc sai định dạng hoặc vượt quá dung lượng cho phép, hệ thống thông báo lỗi. \newline
  5a. Nếu kết nối internet bị gián đoạn trong khi tải lên, hệ thống hiển thị thông báo thất bại và yêu cầu thử lại. \newline
  6a. Nếu hệ thống lỗi trong quá trình xử lý file, hiển thị thông báo “Upload thất bại”. \\ \hline
\textbf{Alternative Flow} 
& 3b. Người dùng có thể hủy thao tác upload bất kỳ lúc nào để quay lại thư viện. \newline
  4b. Người dùng có thể chọn chế độ chia sẻ: công khai, riêng tư. \newline
  5b. Hệ thống hỗ trợ upload nhiều bài hát cùng lúc để tiết kiệm thời gian. \\ \hline
\end{tabularx}
\caption{Mô tả usecase Upload bài hát/album}
\end{table}

% for testing
% REFERENCES
\pagebreak
\nocite{*}
\printbibliography[
heading=bibintoc,
title={References}
]
\begin{thebibliography}{9}

\bibitem{gilmore1961}
Gilmore, P.C. and Gomory, R.E., 1961. A linear programming approach to the cutting stock problem. \textit{Operations Research}, 9(6), pp.849-859.

\bibitem{dyckhoff1991}
Dyckhoff, H., 1991. Cutting stock problems and solution procedures. \textit{European Journal of Operational Research}, 44(2), pp.145-159.

\bibitem{cui2000}
Cui, Z. and Zhang, L., 2000. A near-optimal solution to a two-dimensional cutting stock problem. \textit{Journal of Optimization Theory and Applications}, 107(2), pp.393-408.

\bibitem{bennell2021}
Bennell, J.A., Oliveira, J.F. and Hitchen, M., 2021. Exact solution techniques for two-dimensional cutting and packing. \textit{European Journal of Operational Research}, 293(3), pp.949-963.

\bibitem{lodi2021}
Lodi, A., Martello, S. and Vigo, D., 2021. A heuristic approach for two-dimensional rectangular cutting. \textit{Computers \& Operations Research}, 127, p.105121.

\bibitem{silva2023}
Silva, R.A., Pinto, T.L. and Gomes, C.J., 2023. Approximation method for solving two-dimensional cutting stock problems. \textit{Mathematical Programming}, 150(1), pp.195-220.

\bibitem{vanderbeck1996}
Vanderbeck, F. and Wolsey, L.A., 1996. An exact algorithm for the two-dimensional cutting stock problem. \textit{Computational Optimization and Applications}, 3(1), pp.123-143.

\bibitem{alvarez2005}
Alvarez-Valdes, R., Parajon, A. and Tamarit, J.M., 2005. A tabu search algorithm for large-scale two-dimensional cutting stock problems. \textit{Computers \& Operations Research}, 32(5), pp.985-1007.

\bibitem{beasley1985}
Beasley, J.E., 1985. An exact two-dimensional non-guillotine cutting tree search procedure. \textit{Operations Research}, 33(1), pp.49-64.

\bibitem{martello1990}
Martello, S. and Toth, P., 1990. Knapsack Problems: Algorithms and Computer Implementations. Wiley-Interscience.

\bibitem{gfgGreedyAlgorithm}
GeeksforGeeks, 2024. Introduction to Greedy Algorithm - Data Structures and Algorithm Tutorials. Available at: \url{https://www.geeksforgeeks.org/introduction-to-greedy-algorithm-data-structures-and-algorithm-tutorials/} [Accessed 25 Nov. 2024].

\bibitem{orlibrary}
OR-Library, Cutting Stock Problem Data. Available at: \url{http://people.brunel.ac.uk/~mastjjb/jeb/orlib/cutinfo.html} [Accessed 25 Nov. 2024].

\bibitem{dyckhoff1990}
Dyckhoff, H., 1990. A typology of cutting and packing problems. \textit{European Journal of Operational Research}, 44(2), pp.145-159.

\end{thebibliography}
\end{document}