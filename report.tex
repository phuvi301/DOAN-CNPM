% FORMAT AND PACKAGES
% {
\documentclass[a4paper]{article}
\usepackage{a4wide,amssymb,epsfig,latexsym,multicol,array,hhline,fancyhdr}
\usepackage{tcolorbox}
\usepackage{minted}
\usepackage{vntex}
\usepackage{amsmath}
\usepackage{lastpage}
\usepackage[lined,boxed,commentsnumbered]{algorithm2e}
\usepackage{enumerate}
\usepackage{xcolor}
\usepackage{graphicx}							% Standard graphics package
\usepackage{array}
\usepackage{tabularx, caption}
\usepackage{multirow}
\usepackage{multicol}
\usepackage{rotating}
\usepackage{graphics}
\usepackage{geometry}
\usepackage{setspace}
\usepackage{epsfig}
\usepackage{tikz}
\usepackage{xfrac}
\usepackage{bm}
\usepackage{biblatex}
\usepackage[colorlinks]{hyperref}
\newcommand{\cach}{\hspace*{1.5em}\ignorespaces}
% \usepackage[acronym,toc]{glossaries}
% \usepackage[symbols,nogroupskip,nonumberlist]{glossaries-extra}
\usepackage[
 sort=none,% no sorting or indexing required
 abbreviations,% create list of abbreviations
 symbols,% create list of symbols
 stylemods,style=list, % set the default glossary style
 nogroupskip, nonumberlist, nomain
]{glossaries-extra}


% FORMATTING
% {
\DeclareMathOperator{\arccot}{arccot}
\captionsetup[table]{name=Bảng}
\captionsetup[figure]{name=Hình}
\newenvironment{Description}{\list{}{%
    \let\makelabel\descriptionlabel    % this comes from the original description environment
    \setlength{\rightmargin}{\leftmargin}% this comes from the original quote environment
    \setlength{\labelwidth}{0pt}%          this is new
    }}{\endlist}

\addbibresource{citations.bib}
    
\hypersetup{urlcolor=blue,linkcolor=black,citecolor=black,colorlinks=true} 
\usetikzlibrary{arrows,snakes,backgrounds}
\definecolor{mathblue}{RGB}{0,114,188}
% \makeatletter  \def\m@th{\mathsurround\z@\color{mathblue}} \makeatother
% \everymath{\color{mathblue}}
% \setmathfont[Color=000000]{Arial}
%\usepackage{pstcol} 								% PSTricks with the standard color package
\newtheorem{theorem}{{\bf Theorem}}
\newtheorem{property}{{\bf Property}}
\newtheorem{proposition}{{\bf Proposition}}
\newtheorem{corollary}[proposition]{{\bf Corollary}}
\newtheorem{lemma}[proposition]{{\bf Lemma}}

\AtBeginDocument{\renewcommand{\listfigurename}{Danh sách hình ảnh}}
\AtBeginDocument{\renewcommand{\listtablename}{Danh sách bảng biểu}}
\AtBeginDocument{\renewcommand*\contentsname{Mục lục}}
\AtBeginDocument{\renewcommand*\refname{Tài liệu tham khảo}}
%\usepackage{fancyhdr}

\setlength{\headheight}{40pt}
\pagestyle{fancy}
\fancyhead{} % clear all header fields
\fancyhead[L]{
 \begin{tabular}{rl}
    \begin{picture}(25,15)(0,0)
    \put(0,-8){\includegraphics[width=8mm, height=8mm]{hcmut.png}}
    %\put(0,-8){\epsfig{width=10mm,figure=hcmut.eps}}
   \end{picture}&
	%\includegraphics[width=8mm, height=8mm]{hcmut.png} & %
	\begin{tabular}{l}
		\textbf{\bf \ttfamily Trường Đại học Bách Khoa TP.Hồ Chí Minh}\\
		\textbf{\bf \ttfamily Khoa Khoa học và Kỹ thuật máy tính}
	\end{tabular} 	
 \end{tabular}
}
\fancyhead[R]{
	\begin{tabular}{l}
		\tiny \bf \\
		\tiny \bf 
	\end{tabular}  }
\fancyfoot{} % clear all footer fields
\fancyfoot[L]{\scriptsize \ttfamily Báo cáo Bài tập lớn môn Mô hình hóa toán học - HK241 - Năm học 2024 - 2025}
\fancyfoot[R]{\scriptsize \ttfamily Trang {\thepage}/\pageref{LastPage}}
\renewcommand{\headrulewidth}{0.3pt}
\renewcommand{\footrulewidth}{0.3pt}

\setcounter{secnumdepth}{4}
\setcounter{tocdepth}{4}

\makeatletter
\newcounter {subsubsubsection}[subsubsection]
\renewcommand\thesubsubsubsection{\thesubsubsection .\@alph\c@subsubsubsection}
\newcommand\subsubsubsection{\@startsection{subsubsubsection}{4}{\z@}%
                                     {-3.25ex\@plus -1ex \@minus -.2ex}%
                                     {1.5ex \@plus .2ex}%
                                     {\normalfont\normalsize\bfseries}}
\newcommand*\l@subsubsubsection{\@dottedtocline{3}{10.0em}{4.1em}}
\newcommand*{\subsubsubsectionmark}[1]{}
% \def\m@th{\mathsurround\z@\color{mathblue}}
\makeatother
% }
% }

% ACRONYMS & SYMBOLS
% {
% \makeglossaries
\setabbreviationstyle{long-short}
\newabbreviation{csp}{CSP}{Cutting Stock Problem}
\newabbreviation{ffd}{FFD}{First Fit Decreasing}
\newabbreviation{ga}{GA}{Genetic Algorithm}
\newabbreviation{lp}{LP}{Linear Programming}
% \glsnoexpandfields
\glsxtrnewsymbol[description = {Tập hợp số tự nhiên}]{natural}{\ensuremath{\mathbb{N}}}

% }
%
% DOCUMENT
\begin{document}

% TITLE PAGE
\begin{titlepage}
\begin{center}
ĐẠI HỌC QUỐC GIA THÀNH PHỐ HỒ CHÍ MINH\\
TRƯỜNG ĐẠI HỌC BÁCH KHOA\\
KHOA KHOA HỌC VÀ KỸ THUẬT MÁY TÍNH\\
\end{center}

\vspace{1cm}

\begin{figure}[h!]
\begin{center}
\includegraphics[width=3cm]{hcmut.png}
\end{center}
\end{figure}

\vspace{1cm}


\begin{center}
\begin{tabular}{c}
\multicolumn{1}{c}{\textbf{{\Large Đồ án tổng hợp - CNPM (CO3103)}}}\\
~~\\
\hline
\\
\multicolumn{1}{l}{\textbf{{\Large Bài tập lớn}}}\\
\\
\textbf{\textit{{\Huge "Ứng dụng nghe nhạc trực tuyến"}}}\\
\\
\hline
\end{tabular}
\end{center}

\vspace{2cm}

\begin{table}[h]
\centering
    \begin{tabular}{rl}
    \hspace{3 cm}\textbf{GVHD}:
    & ThS. Trần Trương Tuấn Phát\\

    & \\[10pt]
\textbf{Sinh viên}: & Lư Chấn Vũ - 2313955 \emph{(Nhóm 10, \textbf{Leader})} \\
& Nguyễn Phú Vinh - 2313922 \emph{(Nhóm 10)} \\
& Trần Dương Khiết Nhi - 2312509 \emph{(Nhóm 10)} \\
& Lê Minh Khoa - 2311593 \emph{(Nhóm 10)} \\
& Lê Minh Trí - 2313593 \emph{(Nhóm 10)} \\
    \end{tabular}
\end{table}

\begin{center}
{\footnotesize TP. HỒ CHÍ MINH, 09/2025}
\end{center}
\end{titlepage}

\pagebreak
\tableofcontents

\pagebreak

% Glossaries
% {}
\printunsrtglossary[type={symbols}, title={Danh sách kí hiệu}]
\printunsrtglossary[type={abbreviations}, title={Danh sách từ viết tắt}]
\pagebreak
\listoffigures
\listoftables
\pagebreak
\addcontentsline{toc}{section}{\listfigurename}
\addcontentsline{toc}{section}{\listtablename}

% 

% Member list
\section*{Danh sách thành viên và nhiệm vụ}
\addcontentsline{toc}{section}{Danh sách thành viên và nhiệm vụ}
\begin{center}
\begin{table}[H]
\centering
\begin{tabular}{|c|c|c|l|c|}
\hline
\textbf{STT} & \textbf{Họ và tên} & \textbf{MSSV} & \textbf{Nhiệm vụ} & \textbf{\% hoàn thành}\\
\hline 
%%%%%Student 1%%%%%%%%%%
\multirow{3}{*}{1} & \multirow{3}{*}{Lư Chấn Vũ} & \multirow{3}{*}{2313955} & 
- & \multirow{3}{*}{100\%}\\
 & & & - & \\
\hline
%%%%%Student 2%%%%%%%%%%
\multirow{3}{*}{2} & \multirow{3}{*}{Nguyễn Phú Vinh} & \multirow{3}{*}{2313922} & 
- & \multirow{3}{*}{100\%}\\
 & & & - & \\
\hline
%%%%%Student 3%%%%%%%%%%
\multirow{3}{*}{3} & \multirow{3}{*}{Trần Dương Khiết Nhi} & \multirow{3}{*}{2312509} & 
-& \multirow{3}{*}{100\%}\\
 & & & - & \\
\hline
%%%%%Student 4%%%%%%%%%%
\multirow{3}{*}{4} & \multirow{3}{*}{Lê Minh Khoa} & \multirow{3}{*}{2311593} & 
- & \multirow{3}{*}{100\%}\\
 & & & - & \\
\hline
%%%%%Student 5%%%%%%%%%%
\multirow{3}{*}{5} & \multirow{3}{*}{Lê Minh Trí} & \multirow{3}{*}{2313593} & 
- & \multirow{3}{*}{100\%}\\
 & & & - & \\
\hline
\end{tabular}
\caption{\label{table1}Danh sách thành viên và nhiệm vụ}
\end{table}
\end{center}

\pagebreak
\section*{Nội dung báo cáo}

% Quản lý tài khoản - Người dùng
\begin{table}[h!]
\centering
\renewcommand{\arraystretch}{1.3} % tăng khoảng cách dòng trong bảng
\begin{tabularx}{\textwidth}{|l|X|}
\hline
\textbf{Use case name} & Quản lý tài khoản \\ \hline
\textbf{Created by}    & Vinh \\ \hline
\textbf{Actors}        & Người Dùng \\ \hline
\textbf{Description}   & Nơi để Người Dùng có thể quản lý tài khoản của mình (cập nhật thông tin cá nhân, đổi mật khẩu, đăng nhập, đăng ký, đăng xuất). \\ \hline
\textbf{Trigger}       & Người Dùng chọn chức năng “Quản lý tài khoản” trên giao diện hệ thống sau khi đăng nhập thành công. \\ \hline
\textbf{Pre-Condition(s)} 
& 1. Thiết bị của Người Dùng phải được kết nối internet. \newline
  2. Hệ thống hoạt động bình thường. \\ \hline
\textbf{Post-Condition(s)} 
& Thông tin tài khoản được cập nhật và lưu trữ an toàn trong hệ thống. \\ \hline
\textbf{Normal Flow}   
& 1. Người Dùng chọn vào phần “Quản lý tài khoản” trong giao diện hệ thống. \newline
  2. Hệ thống hiển thị thông tin tài khoản hiện tại. \newline
  3. Người Dùng chọn hành động: chỉnh sửa thông tin cá nhân / đổi mật khẩu / thiết lập bảo mật. \newline
  4. Người Dùng nhập thông tin mới hoặc thay đổi cần thiết. \newline
  5. Người Dùng xác nhận và lưu thay đổi. \newline
  6. Hệ thống cập nhật dữ liệu và thông báo thành công. \\ \hline
\textbf{Exception Flow} 
& 3a. Nếu hệ thống không tải được thông tin tài khoản thì hiển thị lỗi “Không thể tải dữ liệu”. \newline
  4a. Nếu thông tin nhập sai định dạng (ví dụ email không hợp lệ, mật khẩu quá ngắn), hệ thống thông báo lỗi và yêu cầu nhập lại. \newline
  5a. Nếu kết nối internet bị gián đoạn trong khi lưu, hệ thống hiển thị thông báo thất bại và yêu cầu thử lại. \\ \hline
\textbf{Alternative Flow} 
& 3b. Người Dùng có thể hủy thao tác và quay lại trang chính mà không thay đổi gì. \newline
  4b. Người Dùng có thể bật/tắt các tính năng nâng cao như xác thực 2 lớp. \newline
  5b. Người Dùng có thể tải xuống bản sao dữ liệu tài khoản của mình để lưu trữ. \\ \hline
\end{tabularx}
\caption{Mô tả usecase Quản lý tài khoản}
\end{table}

% Tương tác - Người dùng
\begin{table}[h!]
\centering
\renewcommand{\arraystretch}{1.3} % tăng khoảng cách dòng trong bảng
\begin{tabularx}{\textwidth}{|l|X|}
\hline
\textbf{Use case name} & Tương tác \\ \hline
\textbf{Created by}    & Vinh \\ \hline
\textbf{Actors}        & Người Dùng \\ \hline
\textbf{Description}   & Người Dùng có thể thực hiện các hành động tương tác trong hệ thống như: thích (like) bài hát, bình luận bài hát, theo dõi nghệ sĩ, chia sẻ bài hát/nghệ sĩ/playlist với bạn bè hoặc lên mạng xã hội. \\ \hline
\textbf{Trigger}       & Người Dùng chọn một bài hát, nghệ sĩ hoặc playlist bất kỳ trong hệ thống và mở giao diện chi tiết. \\ \hline
\textbf{Pre-Condition(s)} 
& 1. Thiết bị của Người Dùng phải được kết nối internet. \newline
  2. Người Dùng đã đăng nhập thành công. \\ \hline
\textbf{Post-Condition(s)} 
& Hệ thống lưu lại thông tin tương tác (like, bình luận, theo dõi, chia sẻ) và cập nhật dữ liệu liên quan. \\ \hline
\textbf{Normal Flow}   
& 1. Người Dùng chọn một bài hát/nghệ sĩ/playlist. \newline
  2. Hệ thống hiển thị chi tiết bài hát/nghệ sĩ/playlist. \newline
  3. Người Dùng chọn hành động muốn thực hiện: \newline
  \cach 3a. Nhấn nút “Thích” (Like) để thêm vào danh sách yêu thích. \newline
  \cach 3b. Viết và gửi bình luận cho bài hát. \newline
  \cach 3c. Nhấn “Theo dõi” để theo dõi nghệ sĩ. \newline
  \cach 3d. Chọn “Chia sẻ” và lựa chọn kênh chia sẻ (bạn bè trong ứng \cach dụng, mạng xã hội). \newline
  4. Hệ thống xác nhận và cập nhật thông tin tương tác. \\ \hline
\textbf{Exception Flow} 
& 2a. Nếu bài hát/nghệ sĩ/playlist không tồn tại hoặc bị xóa $\rightarrow$ hiển thị thông báo lỗi. \newline
  3b1. Nếu bình luận chứa nội dung vi phạm chính sách $\rightarrow$ hệ thống từ chối đăng và hiển thị thông báo. \newline
  3d1. Nếu chia sẻ thất bại do mất kết nối internet $\rightarrow$ hiển thị thông báo “Chia sẻ không thành công, vui lòng thử lại”. \\ \hline
\textbf{Alternative Flow} 
& 3a1. Người Dùng có thể “Bỏ thích” (Unlike) nếu trước đó đã thích. \newline
  3b2. Người Dùng có thể chỉnh sửa hoặc xóa bình luận đã đăng. \newline
  3c1. Người Dùng có thể hủy theo dõi nghệ sĩ. \newline
  3d2. Người Dùng có thể sao chép đường dẫn (link) thay vì chia sẻ trực tiếp. \\ \hline
\end{tabularx}
\caption{Mô tả usecase Tương tác}
\end{table}

% Nghe nhạc - Người dùng
\begin{table}[h!]
\centering
\renewcommand{\arraystretch}{1.3} % tăng khoảng cách dòng trong bảng
\begin{tabularx}{\textwidth}{|l|X|}
\hline
\textbf{Use case name} & Nghe nhạc \\ \hline
\textbf{Created by}    & Vinh \\ \hline
\textbf{Actors}        & Người Dùng \\ \hline
\textbf{Description}   & Người Dùng có thể phát và điều khiển nhạc với nhiều tính năng nâng cao: phát/tạm dừng, tiếp tục, chuyển tiếp/lùi lại, bật chế độ nghe ngẫu nhiên, nghe lặp lại (một bài hoặc toàn bộ danh sách), thêm bài vào hàng chờ, cài giờ tắt nhạc (sleep timer), và xem lời bài hát. \\ \hline
\textbf{Trigger}       & Người Dùng chọn một bài hát/playlist/album và nhấn nút “Phát nhạc”. \\ \hline
\textbf{Pre-Condition(s)} 
& 1. Thiết bị của Người Dùng phải được kết nối internet. \newline
  2. Ứng dụng được cấp quyền truy cập âm thanh. \\ \hline
\textbf{Post-Condition(s)} 
& Hệ thống phát nhạc theo thao tác của Người Dùng, cập nhật trạng thái trình phát (player state) và lưu lại lịch sử nghe nhạc. \\ \hline
\textbf{Normal Flow}   
& 1. Người Dùng chọn một bài hát/playlist/album. \newline
  2. Hệ thống tải dữ liệu nhạc và bắt đầu phát. \newline
  3. Người Dùng có thể thực hiện các thao tác điều khiển: \newline
  \cach 3a. Nhấn “Tạm dừng” (Pause) hoặc “Tiếp tục” (Play). \newline
  \cach 3b. Nhấn “Chuyển tiếp” (Next) hoặc “Lùi lại” (Previous). \newline
  \cach 3c. Bật chế độ “Nghe ngẫu nhiên” (Shuffle). \newline
  \cach 3d. Chọn chế độ “Nghe lặp lại” (Repeat one / Repeat all). \newline
  \cach 3e. Thêm bài hát vào hàng chờ phát (Queue). \newline
  \cach 3f. Cài đặt “Giờ tắt nhạc” (Sleep timer). \newline
  \cach 3g. Xem lời bài hát (Lyrics) nếu có sẵn. \newline
  4. Hệ thống phản hồi ngay lập tức và cập nhật trình phát nhạc. \\ \hline
\textbf{Exception Flow} 
& 2a. Nếu bài hát không thể phát (do bản quyền hoặc lỗi file) $\rightarrow$  hiển thị thông báo. \newline
  3f1. Nếu Người Dùng cài giờ tắt nhạc nhưng app bị thoát trước thời điểm đó $\rightarrow$ hệ thống không thể tắt nhạc đúng hẹn. \newline
  3g1. Nếu bài hát không có lời (lyrics) trong cơ sở dữ liệu $\rightarrow$ hiển thị “Chưa có lời bài hát”. \\ \hline
\textbf{Alternative Flow} 
& 3a1. Người Dùng có thể sử dụng tai nghe hoặc thiết bị ngoài để điều khiển (nút Play/Pause). \newline
  3e1. Người Dùng có thể sắp xếp lại thứ tự bài hát trong hàng chờ. \newline
  3f2. Người Dùng có thể hủy hoặc thay đổi thời gian sleep timer. \\ \hline
\end{tabularx}
\caption{Mô tả usecase Nghe nhạc}
\end{table}

% Khám phá và tìm kiếm - Người dùng
\begin{table}[h!]
\centering
\renewcommand{\arraystretch}{1.3} % tăng khoảng cách dòng trong bảng
\begin{tabularx}{\textwidth}{|l|X|}
\hline
\textbf{Use case name} & Khám phá và tìm kiếm \\ \hline
\textbf{Created by}    & Vinh \\ \hline
\textbf{Actors}        & Người Dùng \\ \hline
\textbf{Description}   & Người Dùng có thể tìm kiếm bài hát, nghệ sĩ, album, playlist và khám phá nhạc mới thông qua bảng xếp hạng, xu hướng (trending) và gợi ý cá nhân hóa. \\ \hline
\textbf{Trigger}       & Người Dùng mở thanh tìm kiếm hoặc tab “Khám phá” trong ứng dụng. \\ \hline
\textbf{Pre-Condition(s)} 
& Thiết bị của Người Dùng phải được kết nối internet. \\ \hline
\textbf{Post-Condition(s)} 
& Hệ thống trả về kết quả tìm kiếm hoặc danh sách nhạc khám phá phù hợp, cho phép Người Dùng chọn và nghe nhạc ngay. \\ \hline
\textbf{Normal Flow}   
& 1. Người Dùng mở tính năng Tìm kiếm/Khám phá. \newline
  2. Người Dùng có thể thực hiện: \newline
  \cach 2a. Nhập từ khóa để tìm bài hát, nghệ sĩ, album, playlist. \newline
  \cach 2b. Xem danh sách Trending hoặc Top Chart. \newline
  \cach 2c. Nhận gợi ý nhạc cá nhân hóa dựa trên lịch sử nghe, lượt thích và nghệ sĩ theo dõi. \newline
  3. Hệ thống hiển thị danh sách kết quả. \newline
  4. Người Dùng chọn nội dung mong muốn (phát nhạc, xem chi tiết nghệ sĩ/album/playlist). \\ \hline
\textbf{Exception Flow} 
& 2a1. Nếu không tìm thấy kết quả $\rightarrow$ hiển thị thông báo “Không tìm thấy nội dung phù hợp”. \newline
  2b1. Nếu hệ thống chưa có dữ liệu trending/top chart $\rightarrow$ hiển thị “Dữ liệu đang cập nhật”. \newline
  2c1. Nếu Người Dùng mới/ẩn danh (không đăng nhập) chưa có lịch sử nghe $\rightarrow$ gợi ý nhạc phổ biến mặc định. \\ \hline
\textbf{Alternative Flow} 
& 2a2. Người Dùng có thể dùng bộ lọc nâng cao (theo thể loại, thời lượng, năm phát hành…). \newline
  2b2. Người Dùng có thể chọn xem bảng xếp hạng theo từng khu vực/quốc gia. \newline
  2c2. Người Dùng có thể cập nhật gợi ý bằng cách thay đổi sở thích cá nhân (genres, mood…). \\ \hline
\end{tabularx}
\caption{Mô tả usecase Khám phá và tìm kiếm}
\end{table}

\begin{table}[h!]
\centering
\renewcommand{\arraystretch}{1.3} % tăng khoảng cách dòng trong bảng
\begin{tabularx}{\textwidth}{|l|X|}
\hline
\textbf{Use case name} & Quản lý danh sách nhạc \\ \hline
\textbf{Actors}        & Người Dùng \\ \hline
\textbf{Description}   & Cho phép người dùng xem, thêm, sửa và xóa bài hát trong danh sách nhạc của họ. \\ \hline
\textbf{Trigger}       & Người dùng truy cập danh sách nhạc và chọn một chức năng quản lý danh sách nhạc (tạo danh sách, thêm bài hát, xóa bài hát, xóa danh sách, .v.v). \\ \hline
\textbf{Pre-Condition(s)} 
& 1. Thiết bị của Người dùng phải được kết nối internet. \newline
  2. Người dùng đã đăng nhập thành công. \\ \hline
\textbf{Post-Condition(s)} 
& 1. Danh sách nhạc của User được cập nhật theo hành động đã thực hiện (thêm, sửa, xóa). \newline
  2. Người dùng nhận được thông báo về kết quả của hành động. \\ \hline
\textbf{Normal Flow}   
& 1. Người dùng truy cập vào phần “Danh sách nhạc của tôi” trong giao diện hệ thống. \newline
  2. Hệ thống hiển thị danh sách nhạc của người dùng hiện tại. \newline
  3. Người dùng thực hiện thao tác điều khiển: \newline
  \cach 3a. Người dụng chọn ``Tạo playlist mới'', hệ thống hiển thị yêu cầu điền tên playlist, người dùng nhập và nhấn xác nhận. \newline
  \cach 3b. Người dùng chọn ``Thêm vào playlist'' với bài hát muốn thêm, hệ thống yêu cầu chọn playlist muốn thêm, người dùng chọn và nhấn xác nhận. \newline
  \cach 3c. Người dùng chọn ``Xóa khỏi playlist'' với bài hát muốn xóa, hệ thống hiển thị thông báo xác nhận, người dùng xác nhận. \newline
  \cach 3d. Người dùng chọn ``Xóa playlist'', hệ thống hiển thị thông báo xác nhận, người dùng xác nhận. \newline
  4. Hệ thống ghi nhận và cập nhật danh sách playlist của người dùng. \\ \hline
\textbf{Exception Flow} 
& 1. Nếu lỗi xảy ra (mất kết nối máy chủ, hệ thống bận, .v.v), mọi yêu cầu từ người dùng đều bị từ chối và hệ thống sẽ hiển thị thông báo ``Server is under maintenance''. \newline
  3a. Nếu người dùng nhập tên playlist đã tồn tại, hệ thống thông báo lỗi và yêu cầu người dùng nhập lại. \newline
  3b. Nếu người dùng thêm 1 bài hát đã tồn tại trong playlist, hệ thống hiển thị thông báo đã tồn tại và không cập nhật bài hát. \\ \hline
\textbf{Alternative Flow} 
& 3a. Người dùng không đặt tên mà trực tiếp xác nhận, hệ thống vẫn sẽ ghi nhận và đặt tên theo format mặc định ``Unname Playlist''.\newline
  3c/3d. Nếu người dùng không xác nhận, hệ thống hoàn tác toàn bộ thao tác và cập nhật lại trạng thái ban đầu của danh sách playlist. \\ \hline
\end{tabularx}
\caption{Mô tả usecase Quản lý danh sách nhạc}
\end{table}

\begin{table}[h!]
\centering
\renewcommand{\arraystretch}{1.3} % tăng khoảng cách dòng trong bảng
\begin{tabularx}{\textwidth}{|l|X|}
\hline
\textbf{Use case name} & Quản trị hệ thống \\ \hline
\textbf{Actors}        & Quản lý \\ \hline
\textbf{Description}   & Cho phép quản lý thực hiện các tác vụ quản trị để duy trì và điều hành hệ thống. \\ \hline
\textbf{Trigger}       & Quản lý truy cập vào trang quản lý hệ thống trong giao diện quản lý. \\ \hline
\textbf{Pre-Condition(s)} 
& 1. Thiết bị của quản lý phải được kết nối internet. \newline
  2. Quản lý đã đăng nhập thành công \textbf{dưới quyền quản trị}. \\ \hline
\textbf{Post-Condition(s)} 
& 1. Các thay đổi về người dùng, nội dung, hoặc bản quyền đã được cập nhật thành công trong hệ thống. \newline
  2. Hệ thống duy trì trạng thái ổn định và các quy định được tuân thủ. \\ \hline
\textbf{Normal Flow}   
& 1. Quản lý truy cập vào trang quản lý hệ thống với tài khoản có quyền quản trị. \newline
  2. Hệ thống điều hướng đến trang quản trị hệ thống. \newline
  3. Quản lý có thể lựa chọn hành động: \newline
  \cach 3a. Xem, sửa (đối với quyền truy cập), xóa người dùng. \newline
  \cach 3b. Phê duyệt, gỡ bỏ nội dung, bài hát, album công khai thủ công. \newline
  \cach 3c. Thay đổi các thông tin của nội dung phù hợp với chính sách. \newline
  \cach 3d. Kiểm tra tính minh bạch và bản quyền của nội dung đăng tải. \newline
  4. Hệ thống thông báo thao tác thành công. \newline
  5. Hệ thống tự động ghi log đảm bảo tính minh bạch về sau. \\ \hline
\textbf{Exception Flow} 
& 3.1. Nếu quản lý cố gắng truy cập một chức năng không có quyền, hệ thống sẽ thông báo lỗi ``Bạn không có quyền truy cập chức năng này''. \newline
  3.2. Nếu có lỗi trong quá trình cập nhật dữ liệu (thông tin không hợp lệ, .v.v), hệ thống sẽ hiển thị thông báo lỗi và không thực hiện thay đổi. \newline
  3.3. Nếu lỗi xảy ra (mất kết nối máy chủ, hệ thống bận, .v.v), mọi thông tin thay đổi sẽ được lưu tạm thời và hệ thống sẽ hiển thị thông báo ``Đang trong quá trình bảo trì'' cho tới khi có kết nối lại và tự động thực hiện lại thao tác. \\ \hline
\textbf{Alternative Flow} 
& 1a. Tài khoản không có quyền thì hệ thống tự động điều hướng đến giao diện người dùng bình thường. \newline
  3a/3b/3c/3d. Nếu thao tác không được xác nhận, hệ thống hoàn tác toàn bộ thao tác và cập nhật lại trạng thái ban đầu của nội dung. \\ \hline
\end{tabularx}
\caption{Mô tả usecase Quản trị hệ thống}
\end{table}
\end{document}